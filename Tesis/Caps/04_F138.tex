\chapter{Estudio sobre el acero F138}\label{C:F138}
\graphicspath{{./figs/04_F138/}}

\begin{table}[!htb]
\centering
\caption{Composición del F138}
\label{tab:F138Comp}
\begin{tabular}{|c|c|c|c|c|c|c|c|c|c|c|}
\hline
\rowcolor[HTML]{BBDAFF} 
\textbf{Fe} & \textbf{Cr} & \textbf{Ni} & \textbf{Mo} & \textbf{Mn} & \textbf{Si} & \textbf{Cu} & \textbf{N} & \textbf{C} & \textbf{P} & \textbf{S} \\ \hline
          &    17.33  &   14.31   &   2.79    &   1.79    &  0.30     &   0.09    &   0.079   &  0.015    &   0.022   &   0.002   \\ \hline
\end{tabular}
\end{table}
\section{Estudio de la microestructura por el método CMWP - Revisión}\label{S:F138CMWP}
\section{Estudio de la microestructura por el método de Langford y figuras de polos generalizadas}\label{S:F138LANG}
\section{Estudio de la microestructura por EBSD - Revisión}\label{S:F138EBSD}
\section{Discusión de resultados}\label{S:F138Dis}
\section{Conclusiones}\label{S:F138Conclusiones}
