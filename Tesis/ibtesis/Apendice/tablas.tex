\chapter{Tablas de calor específico, conductividad térmica y eléctrica utilizadas en las simulaciones por elementos finitos}\label{C:tablas}
%\chapterquote{Negociemos Don Inodoro}{Fernando de la R\'{u}a, 2001}
%\chapterquote{Smartness runs in my family.  When I went to school I was so smart my teacher was in my class for five years}{George Burns}
\graphicspath{{figs/Apendice}}
En las secciones siguientes se muestran los datos utilizados en las simulaciones por elementos finitos que se realizaron en el capítulo \ref{C:dise}. En cada tabla se indica además la fuente de información de cada propiedad física. Los datos que aparecen en este apéndice se obtuvieron a partir de digitalizar gráficas con mediciones de calor específico, conductividad térmica o eléctrica, según correspondía, con excepción de las propiedades físicas del zafiro y el calor específico del silicio, que se obtuvieron de las bibliotecas de datos del programa comercial COMSOL MULTIPHYSICS. Estos últimos datos no se muestran en forma de tabla, sino en forma de polinomios funciones de $T$, que es como aparecían en el programa.
\newpage
\section{Propiedades físicas del MgB$_2$}
\subsection*{Calor específico}
\begin{table}[h!]
  \vspace{-0.5cm}
  \centering
  \begin{tabular}{|c|c|}\hline
  	Temperatura [K]	&	Calor específico [J/(Kg K)]	\\ \hline
  	5.693	&	0.753	\\
	8.759	&	0.780	\\
	12.409	&	0.812	\\
	16.350	&	1.549	\\
	19.562	&	1.929	\\
	21.606	&	2.650	\\
	23.796	&	3.372	\\
	26.861	&	5.507	\\
	29.927	&	6.588	\\
	32.409	&	9.772	\\
	35.037	&	12.958	\\
	36.058	&	15.075	\\
	36.496	&	16.133	\\
	36.934	&	17.191	\\
	37.518	&	18.602	\\
	38.248	&	17.554	\\
	38.978	&	16.858	\\
	40.000	&	17.570	\\
	41.022	&	20.390	\\
	42.044	&	21.804	\\
	44.526	&	25.340	\\
	45.985	&	29.921	\\
	48.175	&	34.157	\\
	49.635	&	37.332	\\
	51.533	&	40.511	\\
	52.555	&	43.683	\\
	54.015	&	48.264	\\
	55.475	&	52.493	\\
	56.496	&	56.719	\\
	57.810	&	59.893	\\
	59.124	&	64.121	\\
	60.438	&	67.998	\\
	62.628	&	73.639	\\
	64.088	&	78.572	\\
	65.110	&	84.203	\\
	66.861	&	89.138	\\
	68.613	&	97.235	\\
	71.095	&	108.501	\\
	73.431	&	118.712	\\
	74.307	&	121.180	\\
	76.204	&	129.630	\\ \hline
  \end{tabular}
  %\vspace{-1cm}
  \caption[Tabla con los valores del calor específico del MgB$_2$]{Tabla con los valores del calor específico del MgB$_2$. Datos obtenidos de \cite{Bauer2001}.}
  \label{tab:cmgb2}
\end{table}

\subsection*{Conductividad térmica}
\begin{table}[h!]
  \centering
  \begin{tabular}{|c|c|}\hline
	Temperatura [K]	&	Conductividad térmica [W/(m K)]	\\ \hline
	7,500	&	1,260	\\
	13,750	&	3,900	\\
	20,000	&	6,803	\\
	25,625	&	9,639	\\
	32,500	&	12,411	\\
	36,875	&	14,193	\\
	40,000	&	14,855	\\
	41,250	&	15,515	\\
	46,250	&	16,377	\\
	50,000	&	17,566	\\
	55,625	&	18,626	\\
	61,250	&	19,554	\\
	67,500	&	20,352	\\
	73,750	&	21,084	\\
	78,750	&	21,485	\\
	84,375	&	21,887	\\
	91,875	&	22,358	\\
	100,000	&	22,763	\\
	106,875	&	22,772	\\
	110,625	&	23,172	\\
	118,750	&	23,183	\\
	134,375	&	23,203	\\
	149,375	&	23,157	\\
	159,375	&	23,236	\\
	174,375	&	23,453	\\
	185,625	&	23,600	\\
	194,375	&	23,677	\\
	204,375	&	23,690	\\
	216,250	&	23,969	\\
	230,625	&	24,382	\\
	247,500	&	24,602	\\
	260,625	&	25,146	\\
	277,500	&	25,826	\\
	292,500	&	26,174	\\ \hline
  \end{tabular}
  \caption[Tabla con los valores de la conductividad térmica del MgB$_2$.]{Tabla con los valores de la conductividad térmica del MgB$_2$. Datos obtenidos de \cite{Putti2003}.}
  \label{tab:kmgb2}
\end{table}
\newpage
\subsection*{Conductividad eléctrica}
La conductividad eléctrica del MgB$_2$ se obtuvo a partir de mediciones de resistividad obtenidas de \cite{Nagamatsu2001}. Como la conductividad eléctrica de un superconductor diverge para temperaturas que están por debajo de $T_c$, el estado superconductor se expresa por una conductividad eléctrica varios órdenes de magnitud mayor que la que tiene el material por encima de $T_c$.
\begin{table}[h!]
  \centering
  \begin{tabular}{|c|c|}\hline
	Temperatura [K]	&	Conductividad eléctrica [S/m]	\\ \hline
	4,710	&	100000000000,000	\\
	20,000	&	100000000000,000	\\
	34,960	&	100000000000,000	\\
	36,640	&	144034522,194	\\
	36,970	&	11074675,429	\\
	37,480	&	2486133,590	\\
	37,650	&	1728814,673	\\
	37,820	&	1414401,150	\\
	38,320	&	1363822,959	\\
	41,510	&	1354673,964	\\
	45,040	&	1354288,694	\\
	47,900	&	1349626,019	\\
	51,260	&	1349261,819	\\
	60,340	&	1339676,254	\\
	68,400	&	1330333,874	\\
	74,450	&	1325500,509	\\
	78,490	&	1325078,975	\\
	83,030	&	1316302,139	\\
	87,900	&	1307606,609	\\
	92,440	&	1299058,962	\\
	96,970	&	1294600,609	\\
	99,500	&	1290372,531	\\ \hline
  \end{tabular}
  \caption[Tabla con los valores de la conductividad eléctrica del del MgB$_2$.]{Tabla con los valores de la conductividad eléctrica del del MgB$_2$. Datos obtenidos de \cite{Nagamatsu2001}.}
  \label{tab:smgb2}
\end{table}
\newpage
\section{Propiedades físicas del silicio}
\subsection*{Calor específico}
Como se explicó al principio del capítulo, los valores de calor específico del silicio no se obtuvieron de tablas, sino de las librerías de propiedades físicas del programa COMSOL MULTIPHYSICS. Este programa utiliza polinomios para aproximar los valores de calor específico. El polinomio que utiliza es de la forma $p_0(T) \,=\, A_0\,+\,A_1 \times T\,+\,A_2 \times T^2\,+\,A_3 \times T^3\,+\,A_4 \times T^4\,$, y los coeficientes del polinomio dependen del intervalo de temperatura considerado. En la tabla \ref{tab:csi} se muestran los coeficientes de dicho polinomio para cada intervalo de temperatura.
\begin{table}[h!]
  %\centering
  \hspace{-1.6cm}
  \begin{tabular}{|c|c|c|c|c|c|c|}\hline
$T_{i}\,[\rm K]$	&	$T_{f}\,[\rm K]$	&	$A_0\,[\rm J/(\rm Kg\,K)]$	&	$A_1\,[\rm J/(\rm Kg\,K^2)]$	&	$A_2\,[\rm J/(\rm Kg\,K^3)]$	&	$A_3\,[\rm J/(\rm Kg\,K^4)]$	&	$A_4\,[\rm J/(\rm Kg\,K^5)]$	\\ \hline
1.0	&	7.0	&$	-4.83\,10^{-5}	$&$	7.68\,10^{-5}	$&$	-3.42\,10^{-5}	$&$	2.81\,10^{-4}	$&$	-3.13\,10^{-7}	$\\ \hline
7.0	&	20.0	&$	0.0525	$&$	-0.0396	$&$	0.0100	$&$	-7.81\,10^{-4}	$&$	3.96\,10^{-5}	$\\ \hline
20.0	&	50.0	&$	-1.81	$&$	0.762	$&$	-0.0865	$&$	0.00374	$&$	-3.33\,10^{-5}	$\\ \hline
50.0	&	293.0	&$	-82.9	$&$	2.71	$&$	0.0140	$&$	-7.98\,10^{-5}	$&$	1.08\,10^{-7}	$\\ \hline
293.0	&	900.0	&$	63.0	$&$	3.77	$&$	-0.00695	$&$	5.95\,10^{-6}	$&$	-1.91\,10^{-9}	$\\ \hline
900.0	&	1685.0	&$	769.0	$&$	0.187	$&$	-3.18\,10^{-5}	$&$	0	$&$	0	$\\ \hline
  \end{tabular}
  \caption[Tabla con coeficientes del polinomio utilizado para calcular el calor específico del silicio.]{Tabla con coeficientes del polinomio utilizado para calcular el calor específico del silicio. El polinomio propuesto es de la forma $p_0(T) \,=\, A_0\,+\,A_1 \times T\,+\,A_2 \times T^2\,+\,A_3 \times T^3\,+\,A_4 \times T^4$. Datos obtenidos de la biblioteca de materiales del programa COMSOL MULTIPHYSICS.}
  \label{tab:csi}
\end{table}
\newpage
\subsection*{Conductividad térmica}
\begin{table}[h!]
  \centering
  \begin{tabular}{|c|c|}\hline
	Temperatura [K]	&	Conductividad térmica [$\frac{\rm W}{\rm m\ K}$]	\\ \hline
	6,897	&	0,0264	\\
	8,513	&	0,0481	\\
	10,251	&	0,0765	\\
	13,628	&	0,1292	\\
	16,821	&	0,1868	\\
	19,039	&	0,2226	\\
	21,548	&	0,2602	\\
	23,792	&	0,2926	\\
	26,928	&	0,3229	\\
	30,478	&	0,3495	\\
	35,801	&	0,3715	\\
	41,024	&	0,3655	\\
	52,553	&	0,3338	\\
	62,500	&	0,2873	\\
	75,256	&	0,2291	\\
	88,400	&	0,1724	\\
	101,299	&	0,1373	\\
	116,080	&	0,1033	\\
	158,197	&	0,0541	\\
	221,000	&	0,0289	\\
	316,476	&	0,0164	\\
	415,574	&	0,0115	\\ \hline
  \end{tabular}
  \caption[Tabla con los valores de la conductividad térmica del silicio.]{Tabla con los valores de la conductividad térmica del silicio. Datos obtenidos de \cite{Glassbrenner1964}.}
  \label{tab:ksi}
\end{table}
\newpage
\section{Propiedades físicas del zafiro}
Al igual que para el caso del calor específico del silicio, el calor específico y la  conductividad térmica del zafiro se obtuvieron a partir de polinomios en $T$, y en las tablas subsiguientes se muestran los coeficientes de dichos polinomios. El calor específico se obtuvo a partir de un polinomio del tipo $p_1(T) \,=\, A_0\,+\,A_1 \times T\,+\,A_2 \times T^2\,+\,A_3 \times T^3\,+\,A_4 \times T^4$, mientras que la conductividad térmica se calculó de un polinomio de grado 5, es decir que $\kappa_{\rm zafiro} \ \approx \ p_2(T) \,=\, A_0\,+\,A_1 \times T\,+\,A_2 \times T^2\,+\,A_3 \times T^3\,+\,A_4 \times T^4\,+\,A_5 \times T^5$.
\subsection*{Calor específico}
\begin{table}[h!]
  %\centering
  \hspace{-1.6cm}
  \begin{tabular}{|c|c|c|c|c|c|c|}\hline
$T_{i}\,[\rm K]$	&	$T_{f}\,[\rm K]$	&	$A_0\,[\rm J/(\rm Kg\,K)]$	&	$A_1\,[\rm J/(\rm Kg\,K^2)]$	&	$A_2\,[\rm J/(\rm Kg\,K^3)]$	&	$A_3\,[\rm J/(\rm Kg\,K^4)]$	&	$A_4\,[\rm J/(\rm Kg\,K^5)]$	\\ \hline
10.0	&	60.0	&$	-0.392	$&$	0.0802	$&$	-0.00507	$&$	1.91\,10^{-4}	$&$	1.78\,10^{-17}	$\\ \hline
60.0	&	130.0	&$	30.0	$&$	-1.46	$&$	0.0191	$&$	1.12\,10^{-4}	$&$	-6.09\,10^{-7}	$\\ \hline
130.0	&	300.0	&$	-42.6	$&$	-1.40	$&$	0.0434	$&$	-1.45E\,10^{-4}	$&$	1.55\,10^{-7}	$\\ \hline
300.0	&	810.0	&$	-528	$&$	7.53	$&$	-0.0139	$&$	1.243\,10^{-5}	$&$	-4.33\,10^{-9}	$\\ \hline
810.0	&	2250.0	&$	745	$&$	0.956	$&$	-7.12\,10^{-4}	$&$	2.69\,10^{-7}	$&$	-3.85\,10^{-11}	$\\ \hline
  \end{tabular}
  \caption[Tabla con coeficientes del polinomio utilizado para calcular el calor específico del zafiro.]{Tabla con coeficientes del polinomio utilizado para calcular el calor específico del zafiro. El polinomio propuesto es de la forma $p_1(T) \,=\, A_0\,+\,A_1 \times T\,+\,A_2 \times T^2\,+\,A_3 \times T^3\,+\,A_4 \times T^4$. Datos obtenidos de la biblioteca de materiales del programa COMSOL MULTIPHYSICS.}
  \label{tab:czaf}
\end{table}

\subsection*{Conductividad térmica}
\begin{table}[h!]
  \centering
  %\hspace{-1.6cm}
  \begin{tabular}{|c|c|c|c|c|}\hline
$T_{i}\,[\rm K]$	&	$T_{f}\,[\rm K]$	&	$A_0\,[\rm J/(\rm Kg\,K)]$	&	$A_1\,[\rm J/(\rm Kg\,K^2)]$	&	$A_2\,[\rm J/(\rm Kg\,K^3)]$	\\ \hline
6.0	&	47.0	&$	-5.11	$&$	1.76	$&$	-0.167	$\\ \hline
47.0	&	108.0	&$	-422	$&$	24.9	$&$	-0.385	$ \\ \hline
108.0	&	300.0	&$	316	$&$	-2.90	$&$	0.0125	$\\ \hline
300.0	&	2073.0	&$	75.8	$&$	-0.190	$&$	2.02\,10^{-4}	$\\ \hline
-&-&-&-&-\\ \hline	
$T_{i}\,[\rm K]$	&	$T_{f}\,[\rm K]$	&	$A_3\,[\rm J/(\rm Kg\,K^4)]$	&	$A_4\,[\rm J/(\rm Kg\,K^5)]$	&	$A_5\,[\rm J/(\rm Kg\,K^6)]$ \\ \hline
6.0	&	47.0	&$	0.0127	$&$	-2.59\,10^{-4}	$&$	1.64\,10^{-6} $\\ \hline
47.0	&	108.0	&$	0.00268	$&$	-8.82\,10^{-6}	$&$	1.12\,10^{-8}	$\\ \hline
108.0	&	300.0	&$	-2.77\,10^{-5}	$&$	3.10\,10^{-8}	$&$	1.37\,10^{-11}	$\\ \hline
300.0	&	2073.0	&$	-9.79\,10^{-8}	$&$	1.78\,10^{-11}	$&$	0	$\\ \hline
  \end{tabular}
  \caption[Tabla con coeficientes del polinomio utilizado para calcular la conductividad térmica del zafiro.]{Tabla con coeficientes del polinomio utilizado para calcular la conductividad térmica del zafiro. El polinomio propuesto es de la forma $p_2(T) \,=\, A_0\,+\,A_1 \times T\,+\,A_2 \times T^2\,+\,A_3 \times T^3\,+\,A_4 \times T^4\,+\,A_5 \times T^5$. Datos obtenidos de la biblioteca de materiales del programa COMSOL MULTIPHYSICS.}
  \label{tab:kzaf}
\end{table}
%%%%%%%%%%%%%%%%%%%%%%%%%%%%%%%%%%%%%%%%%%%%%%%%%%%%%%%%%%%%%%%%%%%%%%%%
