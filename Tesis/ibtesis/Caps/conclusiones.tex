\chapter{Conclusiones}\label{C:concl}
%%%%%%%%%%%%%%%%%%%%%%%%%%%%%%%%%%%%%%%%%%%%%%%%%%%%%%%%%%%%%%%%%%%%%%%%
A lo largo de este trabajo se estudió la viabilidad de construir un detector de neutrones utilizando el superconductor MgB$_2$. Se realizaron simulaciones que permitieron estimar la señal, sensibilidad, eficiencia y tiempo de respuesta del detector, analizando la reacción nuclear $^{10}$B$(n,\alpha)^6$Li que tiene una sección eficaz de 3800\,barns para neutrones térmicos, y libera una energía $Q \ \approx \ 2.3$\,MeV. El calor producido por la reacción produce una supresión momentánea de la superconductividad, lo que produce una señal que permite registrar la captura de un neutrón. A partir de simulaciones con el software SRIM y el programa de simulaciones por elementos finitos COMSOL MULTIPHYSICS se estimó que las partículas depositan su energía cinética en una región de MgB$_2$ del orden de los micrones, a partir de lo cual se concluyó que es preciso crecer el MgB$_2$ en forma de film si se quiere tener sensibilidad suficiente para la detección de neutrones. También se concluyó que para poder tener una buena señal ante el evento de captura de un neutrón es preciso diseñar cables que posean un espesor de unos pocos cientos de nanómetros y un ancho de no mucho más de un micrón, lo que implica que una vez logrado el film de MgB$_2$ será necesario llevar a cabo procesos de micromaquinado sobre el mismo. Se observó que un film con un espesor de 200\,nm tiene una eficiencia de detección muy inferior a uno de un espesor mayor, pero que en un film delgado este aspecto negativo se encuentra compensado sobradamente por el incremento observado en la señal y sensibilidad del detector.

Se realizaron simulaciones de elementos finitos en las que se acopló la física del comportamiento térmico y eléctrico del detector, y se observó que si el mismo es operado con corrientes lo suficientemente bajas, la señal y el tiempo de respuesta del detector no se modifican, y que el detector tiene un tiempo de respuesta de algunos nanosegundos cuando el espesor del cable es de 200\,nm. Se llevaron a cabo simulaciones intentando regular la temperatura del detector simplemente variando la tensión aplicada al mismo, pero se encontró que esa no es una opción viable desde el punto de vista práctico, ya que para regular la temperatura del detector en el rango de interés, es preciso aplicar tensiones que llevan a que por el detector circulen corrientes enormes. Esto implica que la construcción del detector va a requerir un mecanismo adicional para controlar la temperatura del mismo. También se concluyó que por razones de estabilidad en el control de temperatura, es conveniente operar al detector a tensión constante, en vez de hacerlo a corriente constante.

Paralelamente al trabajo de las simulaciones se intentó crecer films de MgB$_2$ por medio de dos técnicas diferentes. En la primera se crecieron films de B por evaporación y luego se recoció a los mismos junto con pastillas de MgB$_2$ bulk en ampollas de cuarzo. Se lograron fabricar films de un espesor de algunos cientos de nanómetros, cuyas curvas de magnetización presentan irreversibilidades compatibles con la formación de una fase superconductora. Sin embargo, siguiendo este método no se pudo conseguir fabricar films con una transición superconductora lo suficientemente estrecha como para poder fabricar el detector, lo que probablemente se deba a que el sustrato reacciona con el film debido a las elevadas temperaturas del recocido.

La segunda técnica de crecimiento de films de MgB$_2$ explorada fue el crecimiento directo de films por sputtering a partir de un blanco de MgB$_2$ obtenido comercialmente. Se realizaron estudios de difracción de rayos X en los que no se pudo observar la formación de la fase MgB$_2$. También se estudió la composición de los films crecidos por medio de espectroscopia de rayos X característicos (EDX) y retrodispersión de Rutherford (RBS), y ambos estudios mostraron que los films crecidos son defectuosos en Mg, lo que probablemente sea la causa de que no sean superconductores, tal como se observó en mediciones de magnetización realizadas en un magnetómetro SQUID. Se decidió intentar recocer los films crecidos junto con pastillas de Mg, en busca de aumentar la cantidad de este elemento en los films, utilizando una temperatura de recocido más baja que la empleada con los films crecidos por e\-va\-po\-ra\-ción. Luego del recocido se observó una mejora en las propiedades de transporte de las muestras, ya que pasaron de ser aislantes a semiconductoras, pero no se pudo observar la formación de fases superconductoras, ni en mediciones de magnetización, ni en mediciones de transporte. Esto último se debió probablemente a que la temperatura de recocido no fue lo suficientemente alta como para permitir la difusión de la cantidad necesaria de Mg a través del film.

En lo respectivo a la fabricación de films superconductores se concluyó que para conseguir formar películas delgadas de MgB$_2$ es preciso realizar recocidos a tem\-pe\-ra\-tu\-ras del orden de los 700\,$^{\circ}$C junto con MgB$_2$ bulk, si es que se quiere fabricar los films utilizando este método. La implementación de esta técnica requiere además la utilización de un sustrato que no reaccione con el film a altas temperaturas. Por otro lado, si se quiere evitar el paso de recocido y crecer los films de MgB$_2$ en un solo paso, resulta necesario emplear un método de codepósito de Mg y B, en el que se pueda regular independientemente cuanto de cada material se deposita en el sustrato.
