\begin{resumen}%
  En este trabajo se estudia la viabilidad de utilizar films del supuerconductor MgB$_2$, que tiene una temperatura crítica alrededor de 39\,K, para construir un detector de neutrones térmicos y fríos. El objetivo es aprovechar el calor generado por la reacción $^{10}$B$(n,\alpha)^6$Li que tiene una sección eficaz de 3800\,barns para neutrones térmicos y libera una energía de aproximadamente 2.3\,MeV. El calor producido por la reacción produce una supresión momentánea de la superconductividad, lo que produce una señal que permite registrar la captura de un neutrón. Se realizaron simulaciones de las trayectorias de los productos de la reacción en el MgB$_2$ para estimar las dimensiones del detector que permitan maximizar su señal, sensibilidad y eficiencia, y que minimicen el tiempo de respuesta, al tiempo que se estimó que la energía de la reacción se deposita en un volumen de unos pocos micrómetros cúbicos. Cálculos y simulaciones hechas con el software comercial de elementos finitos COMSOL MULTIPHYSICS llevaron a la conclusión de que para poder detectar eficientemente neutrones con el MgB$_2$, resulta necesario construir un cable que tenga un ancho no mayor a un micrón y un espesor no mucho mayor a los 200\,nm. Dimensiones mayores incrementan la probabilidad de captura de un neutrón pero reducen drásticamente la señal y la sensibilidad del detector, además de que dificultan el control de la temperatura del mismo.

Fueron realizadas simulaciones en las que se acopló la física del comportamiento térmico y eléctrico del detector, y se observó que si el mismo es operado con corrientes lo suficientemente bajas, la señal y el tiempo de respuesta del detector no se modifican, y que tiene un tiempo de respuesta de algunos nanosegundos cuando el espesor del cable es de 200\,nm. También se llevaron a cabo simulaciones intentando regular la temperatura del detector simplemente variando la tensión aplicada al mismo, pero el resultado fue que eso no es posible desde el punto de vista práctico, ya que para regular la temperatura del detector en el rango de interés, es preciso aplicar tensiones que llevan a que circulen corrientes enormes por el cable de MgB$_2$. La conclusión extraída de este cálculo fue que el detector va a requerir un mecanismo adicional para controlar su temperatura. También se concluyó que por razones de estabilidad en el control de la misma, es conveniente operar al detector a tensión constante, en vez de hacerlo a corriente constante.

En conjunto con el trabajo de las simulaciones se intentó crecer films de MgB$_2$ por medio de dos técnicas diferentes, una utilizando un método ex-situ que requrió temperaturas del orden de los 700\,$^{\circ}$C, y otra consistente en un método in-situ que requería temperaturas iban de los de 500\,$^{\circ}$C hasta temperatura ambiente.

La primera técnica de crecimiento consistió en depositar films de B por evaporación para luego recocer los mismos junto con pastillas de MgB$_2$ bulk en ampollas de cuarzo. Se lograron fabricar films de un espesor de algunos cientos de nanómetros, cuyas curvas de magnetización, medidas en un magnetómetro SQUID, presentaron irreversibilidades en un ciclo \textit{Zero Field Cooling - Field Cooling} compatibles con la formación de una fase superconductora. Sin embargo, siguiendo este método no se pudo conseguir fabricar films con una transición superconductora lo suficientemente estrecha como para poder fabricar el detector, lo que probablemente se debió a que el sustrato reaccionaba con el film debido a las elevadas temperaturas del recocido.

La segunda técnica de crecimiento de films de MgB$_2$ explorada en este trabajo fue la de crecimiento directo de films por sputtering, a partir de un blanco de MgB$_2$ obtenido comercialmente. Se realizaron estudios de difracción de rayos X que no mostraron la formación de la fase MgB$_2$. Un estudio de la composición de los films crecidos fue realizado utlizando espectroscopía de rayos X caracterísiticos (EDX) y retrodispersión de Rutherford (RBS). Ambos estudios mostraron que los films crecidos tienen un exceso de B, lo que probablemente sea la causa de que no sean superconductores, tal como mostraron las mediciones de magnetización realizadas sobre las muestras. Se decidió intentar recocer los films crecidos con pastillas de Mg, en busca de mejorar la proporcion B/Mg de los films utilizando una temperatura de recocido más baja que la empleada con los films crecidos por evaporación. El recocido logró una mejora en las propiedades de transporte de las muestras, ya que pasaron de ser aislantes a ser semiconductoras, pero no se pudo observar la formación de fases superconductoras, ni en mediciones de magnetización, ni en mediciones de transporte, lo que consituye un indicio de que los films no lograron incorporar la cantidad suficiente de Mg como para volverse superconductores. Esto último se deba probablemente a que la temperatura de recocido no fue lo suficientemente alta como para permitir la difusión de la cantidad necesaria de Mg a través del film.
\end{resumen}

\begin{abstract}%
	In this paper we study the feasibility of using films of the superconductor MgB$_2$ with a critical temperature of 39\,K, to build a cold neutron and thermal neutron detector. The aim is to use the heat generated by the reaction $^{10}$B$(n,\alpha)^6$Li which has a cross section of 3800\,barns for thermal neutrons and releases an energy of approximately 2.3\,MeV. The heat produced by the reaction causes a partial destruction of superconductivity. One then notices the appearance of a single neutron by the electric resistance variation of the MgB2 thin film. Simulations of the trajectories of the reaction products in the MgB$_2$ were performed for estimating the optimal detector dimensions. Calculations and simulations with the commercial software COMSOL Multiphysics led to the conclusion that in order to efficiently detect neutrons with MgB$_2$, is necessary to build a cable that has a width no greater than one micron and a thickness not much greater than 200\,nm. Larger dimensions increase the probability of neutron capture but drastically reduces the produced signal and the detector sensitivity, plus it difficult the temperature control of the device.

	Simulations were also conducted coupling the physics of the thermal and electrical behavior of the detector, and it was found that if it is operated with a small bias-current, the detector's signal and response time are not changed. Calculations showed that the detector has a response time of few nanoseconds for a wire 200\,nm thick. Simulations were carried out trying to regulate the temperature of the detector by varying the bias tension, but it was found that this is not a viable option from a practical standpoint, as to regulate the temperature of the detector in the range of interest, it is necesary to apply voltages that lead to large currents circulating through the detector. This implied that the construction of the detector will require an additional mechanism to control its temperature. It was also found that for reasons of stability in temperature control, it is desirable to operate the detector voltage-biased instead of current-biased.

	In addition with the simulations, growth of MgB$_2$ films was attempted by two different means, one using an ex-situ method requiring temperatures around 700\,$^{\circ}$C, and an in-situ method requiring temperatures of 500\,$^{\circ}$C down to room temperature.

	The first technique consisted on growing of boron films by evaporation and a post-annealing process with a bulk MgB$_2$ sample in a quartz tube. We managed to make 200\,nm thick films, and perform magnetization measurements in a SQUID magnetoteter. Irreversibilities shown in a Zero Field Coolig - Field Coolig magnetization measurement were compatible with the formation of a superconducting phase. However, it wasn't possible to obtain films with a sharp superconducting transition, as it's needed to build the detector. This is probably the result of a chemical reaction between substrate and the film due to the high annealing temperatures.

  The second technique explored in this work was the direct growth of MgB$_2$ films by sputtering. To this end a commercial MgB$_2$ target was used. Studies performed by X-ray diffraction did not show the formation of the MgB$_2$ crystalline phase. We also studied the composition of the films by electron dispersive X-ray analisys (EDX) and Rutherford backscattering (RBS). Both studies showed that films are grown with an excess of B, which is probably the reason why they are not superconducting, as was observed by magnetization measurements. Films were annealed with Mg pellets, seeking to increase the amount of Mg in the films, using lower annealing temperatures than those used with the films grown by evaporation. Samples showed an improvement in their transport properties after the annealing, as they went from being insulating to semiconductonducting. However no formation of a superconducting phase was observed, neither in magnetization measurements or in electrical resistance measurements. This is probably due to the low annealing temperature, which did not allow the diffusion of the required amount of Mg through the film.
\end{abstract}

%%% Local Variables: 
%%% mode: latex
%%% TeX-master: "template"
%%% End: 